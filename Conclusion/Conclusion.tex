\chapter{Summary and Conclusions}
The aim of this thesis is the development of algorithms for the encoding of the data delivered by the Inner Tracking System of the ALICE experiment during the LHC Run3 that will start in 2020. The ITS-Upgrade consists of monolithic silicon pixel sensors, characterised by a binary readout: the detector provides only spatial information about the crossing point of charged particles within the surface of the sensors.\\
The data format of the ITS-Upgrade is the cluster of pixels, a set of adjacent fired pixels. Two kinds of information are associated to each cluster: the position within the sensor and the shape (topology), i.e. the pattern of fired pixels that constitutes the cluster itself. For each cluster an impact-point position, with the relative uncertainty, is needed. Clusters with the same topology share geometrical information, such as the number of pixels, the dimensions of the bounding box (the smallest rectangle containing entirely the cluster) and the position of the centre of gravity (COG) of the cluster within the bounding box. Moreover, the reconstructed impact-point position (with its uncertainty) within the bounding box is the same for clusters with the same topology.\\
In Run3 there will be an increase of the data-production rate ($\sim$ 40 GB/s for the ITS), due to the high collision rate (50 kHz for Pb-Pb events) and to the number of readout channels ($\sim 10^{10}$). To reduce the data volume, the topologies are stored not as the complete bit-mask corresponding to the active pixels, but as references to a dictionary containing the records about all the topologies. This process of identification of a topology with the corresponding entry in the dictionary is done online during the data acquisition. Due to the high non-uniformity of the distribution of the topologies, these data are compressed using the Huffman coding, which needs to work with at most few thousands of different topologies to be efficient. Therefore, rare topologies with similar number of rows and columns are grouped.\\
The main tool developed in this work is the class Dictionary. The dictionary includes the records of common topologies and groups of rare topologies: the position of the impact point within the bounding box, its uncertainty and the frequency of the topology (or the group).\\ 
The dictionary is built with the class BuildDictionary, starting from the data. This tool collects the information about each topology, such as the impact point of the charged particle that generated the cluster, the frequency of the topology and its geometry. During the construction of the dictionary, for each topology, a distribution of the distances from the impact-point to the COG (residuals) is built. The COG is the geometrical estimator of the impact point and its position within the bounding box is the same for clusters with the same topology. For each topology, its position is set as the mean impact-point position (within the bounding box). The related uncertainty is the standard deviation of the residual distribution. The grouping is performed for topologies that occur with a frequency below a given threshold. For groups of topologies, the impact-point position is the centre of the pixel containing the COG, while its uncertainty is that of a uniform distribution over a range corresponding to the dimensions of the bounding box.\\
The identification of a topology with the corresponding entry in the dictionary is performed online with the class LookUp. This class is based on hash functions: starting from a topology, a eight-byte hashcode is generated. If this hashcode belongs to a common topology, the corresponding entry in the dictionary is found. Otherwise, the hashcode belongs to a rare topology and the group membership is directly computed, finding the corresponding entry in the dictionary. This class has been tested on simulated data and it is fast enough to identify all the topologies in a Pb-Pb event, due to the parallelizability of the problem.\\
Finally, with the class FastSimulation it is possible to reproduce a population of topologies whose frequencies are in agreement with the values within the dictionary. This class, however, does not take into account the characteristics of the particles that produced the topologies. Therefore a few dictionaries are needed, according to the type and the characteristics of the particles that generated the clusters.\\
This work allows also further development, such as the parametrisation of the impact-point uncertainty as a function of the impact angle of the particle or the methods for the creation of specific dictionary according to the characteristics of the particle, which is needed for the fast simulation.