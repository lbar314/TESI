\chapter{Conclusion}
The aim of this thesis was the development of algorithms for the encoding of the data of the ITS-Upgrade, in particular for the encoding of the cluster topology.\\
The main tool developed in this work is the class Dictionary, which includes the information of all the cluster topologies. Thanks to the dictionary, it is possible to store the information about the cluster topology as a reference to the corresponding entry in the database, saving storage space and avoiding to compute many times information that is common for clusters with the same topologies.\\
The dictionary is built with the class BuildDictionary, which creates a dictionary starting from the data. This tool collects the information about all the topologies, such as the impact point, the frequency and the geometry, and groups the rare topologies with similar characteristics.\\
The class LookUp is used online for the identification of a topology with the corresponding entry in the dictionary. It has been tested on simulated data and it is fast enough to identify all the topologies in a Pb-Pb event, due to the parallelizability of the problem.\\
Finally, with the class FastSimulation it is possible to reproduce a population of topologies whose frequencies are in agreement with the values within the dictionary. This class, however, does not take into account the characteristics of the particles that has produced the topologies. Therefore a few dictionaries are needed, according to the type and the characteristics of the particles that generated the clusters.\\
This work allows also further development, such as the parametrisation of the impact-point uncertainty as a function of the impact angle of the particle or the methods for the creation of specific dictionary according to the characteristics of the particle, which is needed for the fast simulation. 