\newenvironment{abstract}%
    {\cleardoublepage\thispagestyle{empty}\null\vfill
    \begin{center}%
      \bfseries\huge\abstractname
    \end{center}}%
    {\vfill\null}
      \begin{abstract}
	ALICE (A Large Ion Collider Experiment) is designed for investigating the nature of the Quark-Gluon Plasma (QGP), a phase of deconfined strongly interacting matter, which occurs in extreme conditions of temperature (T $\sim$ 170 MeV) and baryo-chemical potential ($\mu B \sim$ 0).
	At the Large Hadron Collider, the QGP is obtained through collisions between beams of ultra-relativistic lead nuclei at $\sqrt{s_{NN}}$ = 5.02 TeV, with a collision rate of 8 kHz.\\
	The ALICE apparatus is designed to maximize the reconstruction of the tracks of the particles with low transverse momentum. In order to achieve this objective, detectors characterized by a low read-out rate are used for the Inner Tracking System (ITS). The current ITS configuration has a read-out of 1 kHz and  is not able to process all the Pb-Pb events.\\
	In 2020 there will be un upgrade of the LHC, which will translate in a higher luminosity and it will be possible to collide lead nuclei at the rate of 50 kHz. Thanks to this improvement, it will be possible to collect more statistics and, therefore, to study rare phenomena not accessible at the moment.\\
	For this reason, the ALICE experiment will undergo an upgrade as well, bringing the read-out rate from the current limit up to 50 kHz. In order to fulfill this request, there will be several changes in the detectors. In particular, the ITS, the closest detector to the interaction point and used for the tracking and the determination of the interaction point (vertexing), will be completely substituted.\\
	The ITS-upgrade will consist of seven cylindrical layers of silicon monolithic pixel sensors. These new sensors, characterized by a digital read-out system, have a maximum read-out rate of 100 kHz, making it possible to bear the interaction rate foreseen for the upgrade.\\
	The ITS-upgrade data are in the format of clusters of pixels, i.e. sets of adjacent fired pixels. Two kinds of information are associated to each cluster: its position within the matrix described by the pixels of the sensor and the cluster shape (topology), i.e. the pattern of fired pixels that constitutes the cluster itself.\\
	Starting from a cluster, an impact position with the relative error is needed: clusters with the same topology have the same impact position within the bounding box, i.e. the smallest rectangle containing the cluster itself, and the same related error.\\
	The information about the topology can be stored as a reference to a dictionary containing all the possible topologies, avoiding to store the whole bit-mask corresponding to the cluster shape and allowing to save storage space. Moreover, it is possible to avoid to compute information concerning the topology each time a cluster with a specific topology is found.\\
	Furthermore, Monte Carlo simulations show that the frequency distribution of the topologies is highly not uniform. For this reason, data can be compressed using the Huffman coding, a compression algorithm based on the relative frequency of characters, in this case topologies, which associates the shortest string to the most common character, reducing the storage space needed.\\
	However, the Huffman coding needs to work with at most few thousands of different characters in order to be efficient and for this reason the number of topologies in the dictionary must be reduced. This reduction can be done by grouping rare topologies with similar mean characteristics, in order to reduce the number of entries in the dictionary at the expense of a loss of details about rare instances.\\
	Therefore, the main aim of my thesis work is the development of an algorithm for the creation of a dictionary, containing the information of the common topologies and of the groups of rare topologies. Then, another aim of my project is the implementation of the algorithm for the identification of a topology with the corresponding key in the dictionary, in a time compatible with the read-out rate.\\
      \end{abstract}
