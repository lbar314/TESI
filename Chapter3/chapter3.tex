\chapter{The Upgrade of the ITS}
In its first years of activity, the ALICE experiment has observed the creation of hot hadronic matter at unprecedented values of temperature and density, confirming the nature of the QGP as an almost-perfect fluid, as shown in previous experiments at CERN SPS and at BNL RHIC. Thanks to its unique tracking and PID capabilities, the measurements of the ALICE experiments exceeded the precision and kinematic reach of the QGP measures of its predecessors. However, it is still necessary to carry out high precision measurement of rare probes over a wide $\pt$ range, for which the current setup is not optimised.\\
In 2020 the LHC will increase the luminosity of Pb-Pb collisions, reaching the interaction rate of 50 kHz, corresponding to an instantaneous luminosity of 6 $\mathrm{\times 10^{27} cm^{-2}s^{-1}}$. The ALICE experiment will be consequently upgraded too, in order to enable readout of all the heavy-ion interactions and collecting more than $\mathrm{10 nb^{-1}}$ of Pb-Pb collisions after the upgrade \cite{uptdr}. Among the detectors of the ALICE experiment, the Inner Tracking System will undergo the greatest changes, since it will be replaced with a new detector. In this chapter, the upgraded ITS will be described, focusing on the physics motivations behind the upgrade, the layout of the new detector and the plans for the software development.
\section{Physics Motivations for the Upgrade}
The high capabilities of tracking and identifying particles in a wide $\pt$ range, in an environment characterized by high multiplicities, make ALICE a unique experiment. The side effect of such characteristics is the low readout rate: the current experimental set up does not allow to analyse all the Pb-Pb collisions that occur at the LHC, corresponding to a collision rate of 8 kHz. Therefore, the upgrade program of the ALICE experiment will primarily focus on the increase of the readout rate, to bare the collision rate of the upgraded LHC and be able to analyse all the Pb-Pb interactions. Moreover, with an improved granularity of the detectors, it will be possible to enhance the impact-parameter resolution and to study particular decays that the current apparatus cannot analyse.\\
One of the objective of the upgrade is the study of heavy flavours, which are extremely important in heavy-ion physics since they are mainly produced in the first stages of the collision and therefore allow to study the transport properties of the medium. There are two main questions related to the heavy-flavour interactions in the QGP:
\begin{itemize}
 \item the thermalization of heavy quarks in the medium, studying the baryon/meson ratio for charm ($\Lambda_{c}$/D) and for beauty ($\Lambda_{b}$/B), the azimuthal anisotropy $v_2$ for charm mesons and baryons and possible in-medium thermal production of charm quarks.
 \item the heavy-quark in-medium energy loss and its mass dependence, which can be studied by measuring the nuclear modification factors $\RAA$ of $\pt$ distributions of D and B in a wide momentum range.
\end{itemize}
For all these measurements, it is necessary to reconstruct the open charm and beauty down to low transverse momentum ($\pt \leq$ 1 GeV/c) and to collect large statistics. The current configuration is not sufficient to fulfill these requests and an upgrade of the experimental setup is necessary. In particular, a larger statistics can be collected thanks to the higher collision rate, while the enhancement of high tracking resolution at low $\pt$ can be achieved with a low material budget.\\
Thanks to the reduced material budget and to the improved tracking precision and efficiency, it is possible to carry out a detailed measurement of low-mass dielectrons. These measurements are important because they give access not only to the thermal radiation from the QGP, of both real and virtual photons, but also to the in-medium modification of hadronic spectra related to chiral-symmetry restoration \cite{chiral}. Indeed, the lattice QCD predicts that, at high temperatures, the restoration of chiral symmetry occurs, bringing some distortions in the vector and axial current spectra. This can empirically be seen as a change in the mass spectra, in particular for the meson $\rho$ in it $e^+e^-$ decay mode. This measurement implies the electron detection down to $\sim$ 100 MeV/c and, since the production rates of the thermal dileptons are low, a very good electron identification is needed to suppress combinatorial background, constituted primarily from $\pi^0$ Dalitz decays and photon conversions. For this reason, the upgraded experimental apparatus must be characterized by a low material budget before the first active layer. Moreover, good low-$\pt$ tracking capabilities are needed to allow to track electrons down to $\pt \gtrsim $ 50 MeV/c, improving the reconstruction efficiency for the background suppression. Finally, improving the secondary vertex resolution, it will be possible to discern electrons from semileptonic charm decays and to separate the charm contribution, allowing a better measurement of the thermal radiation.\\
The strong PID capabilities of the ALICE experiment are suitable for the spectroscopy of hypernuclei and exotic objects produced in Pb-Pb collisions. Hypernuclei are nuclei that contain at least a strange baryon (hyperon) in addition to protons and neutrons and their lifetime depend on the strength of the hyperon-nucleon interaction. The importance of such interactions is not only to describe the hadronic phase of a heavy-ion collision, but also to describe the hadronic matter in extreme density conditions, as for instance in neutron stars. In this context hyperon interactions are crucial to understand the phase structure of QCD at large densities. In particular, the mesonic decays of $^{3}_{\Lambda}$H , $^{4}_{\Lambda}$H  and $^{4}_{\Lambda}$He  are studied. The current data allows the detection of these states, but with a poor significance. Therefore, to improve the heavy-nuclear state analyses, the ALICE upgrade must fulfill two requirements: first, it must provide a larger statistics; then, it must improve the separation of the reconstructed signal decays from the combinatorial background, which can be obtained enhancing the tracking resolution. Besides the hypernuclei, more exotic forms of deeply bound states with strangeness have been proposed as states of matter, either consisting of baryons or quarks. Among these exotic objects, there are the H dibaryon, a bound state of two $\Lambda$ mesons, and $\Lambda n$ bound states. These states have not been observed yet and therefore, if they exist, are extremely rare. Moreover, the topology of their decay is very complex, making the detection efficiency low. Similarly to the case of hypernuclei, this analysis would benefit from the larger statistics and from a better tracking resolution.